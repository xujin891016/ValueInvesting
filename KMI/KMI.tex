\documentclass[11pt]{article}
\usepackage{amsmath,amssymb,graphicx}
\usepackage[letterpaper ,left=2cm,top=2.25cm,bottom=2.25cm,right=2.5cm,nohead,nofoot]{geometry}
\usepackage{color}
\definecolor{LinkColor}{rgb}{0.256,0.439,0.588}
\definecolor{MenuColor}{rgb}{0.739,0.394,0.399}
\usepackage[colorlinks,breaklinks,urlcolor=LinkColor]{hyperref}
\newcommand{\menuItem}[1]{{\color{MenuColor}\textsf{#1}}}
\usepackage{bm}
\usepackage{bbold}
\renewcommand{\vec}[1]{\mathbf{#1}}
\DeclareMathOperator{\sgn}{sgn}
\usepackage[T1]{fontenc}
\usepackage{times}
\usepackage[scaled]{beramono}
\begin{document}

\title{Kinder Morgan}
\author{Jin Xu}
\date{\today}
\maketitle

\section{Investment Strategy}

If I were able to fully own a business, I definitely look for one that will consistently generate a large amount of cash. After owning it for a long time, I want to sell it at even a higher price. The investment return would be a combination of cash return and value increase of this business. From the perspective of business, the investment return is the only purpose of owning a business. 

The value of a business should be only determined by its earning capacity in future years. Free cash flow or its other forms should be the best measure of the earning capacity for a stable business. Free cash flow is calculated as {\bf EBIT(1 - Tax Rate) + Depreciation \& Amortization - change in Net Working Capital - Capital Expenditure}. Free cash flow can be distributed to our debt lenders and the business owner. The remaining cash can also be used for business investment, either developing growth projects to increase revenues, or increasing operating efficiency to maximize profit margin. However, it makes sense only if the investment can create more than one dollar for every dollar invested. The free cash flow attributable to business owners should be adjusted by interest payment. As in pipeline business, the widely adopted measure of earning capacity is the distributable cash flow. It is calculated as {\bf Net Income + Depreciation, Depletion \& Amortization - Maintenance Capital Expenditures}. 

The owner does not have to run the business by himself. However, it is necessary for him to fully understand the business. I love simplicity. I even consider simplicity as my religious belief. Actually, a good business can be very simple. As the owner for a long period of time, we need the business to consistently generate stable free cash flow. We can see the consistency from the nature of the business. It should also be confirmed by the track record of its operating history. We should also need to have a positive outlook for its value increase.

Very often, the owner hires a capable executive team to run the business. It is extremely important for the owner to hire good leaders for his company. I strongly believe effective leadership and good management play a way more important role than functional employees. Corporate culture is the spirit of a company as an entity, which helps improve production efficiency and avoid making bad decisions. Effective leaders put emphasis on it. Good management team should always make rational decisions and focus on the long-term value of the company. Also the management team should stay close with the owner and communicate candidly. I believe any environment can be a good opportunity as long as we are prepared and know how to take advantage of it. Good executive team should be vigilant. They should stay fearful when the market is booming, always preparing for the advent of the downturn. When the market bears, they should feel greedy because it is a great opportunity for the company to grow fast at a cheaper price.

In reality, I cannot afford to fully own a good business. I am only able to buy shares of public companies. The stock market is a great platform to buy a tiny portion of a company. If a public company has 1 billion shares outstanding and I only own 100 shares. It simply means I own 1/10000000 portion of the company. I always keep it in mind that owning stocks equals to owning a portion of the company itself. From this philosophy, the stock value should be always associated with a company. However, the market price of the stock can be higher than its value or lower than its value. Thus, my investment strategy is to buy the stock of a good company at a lower-than-value price, own it for a long period, and sell it ultimately at a higher-than-value price. My investment return is the cash return during my holding and the increase in the stock price from my buying to selling.            
 
\section{Good Business}

The oil \& gas industry can be divided into three segments: upstream, midstream and downstream. Upstream produces crude oil and natural gas. This segment bears a large amount of market risk. The performance primarily depends on the global crude oil price. The crude oil price is very volatile by its nature. It is cyclical but extremely difficult for anyone to predict. It suddenly dropped from more than 100 USD per barrel in August, 2014 to less than 30 in early 2016. Midstream stores and transports crude oil and natural gas for processing and processed products to the market. Midstream is the most stable segment among three. It transports oil and gas for its clients or itself and charges them on a fee basis. Downstream refines crude oil and produces petroleum products, manufactures chemical products using products from the upstream, and markets and sells the processed products. The primary market risk is the crack spread. Crack spread is the price difference between crude oil and refined petroleum products. Crack spread varies but not as much as the crude oil price. To be short, midstream is the most stable segment, downstream follows, but upstream is totally volatile.

Kinder Morgan is the largest energy transportation infrastructure company in North America. Almost all of its businesses belong to the midstream segment. The company segments include Natural Gas Pipelines, Carbon Dioxide, Terminals, Product Pipelines, Kinder Morgan Canada and Other. The company builds, expands and acquires midstream assets and then operates on them to generate distributable cash flow. The majority of its cash flow is fee-based and carries small market risk. The distributable cash flow can be regarded as the best measure for its earning capacity. The company returns a portion of distributable cash flow to shareholders through dividend issuances and share buy-backs. Distributable cash flow can be invested on growth, expanding its existing midstream assets or acquires assets from other companies. Finally, distributable cash flow can be used to pay debt principles. It strengthens balance sheet, improves credit rating, reduces interest cost and also helps the organization weather an adverse financial market. It is a simple business which also generates stable distributable cash flow.   

However, this business carries a lot of debts. Both strengthening balance sheet and growing assets are important. The management team should balance weights between two tasks and may shift focus as priority changes. 

\section{Distributable Cash Flow}

With its existing midstream assets, Kinder Morgan can generate around 4.7 billion USD distributable cash flow to common shareholders each year. It may vary due to changes in commodity price, counterparty credit, or interest rates. However, I believe the number should fluctuate within 5\%. 91\% of its cash flow is fee-based, 6\% is hedged, and only 3\% is commodity-based.  The company has a high-quality, diversified customer base. Full-year impact of 100-bp increase in floating rates equates to a pre-tax ~99 million USD increase in interest expense. Due to its fee-based business model, the cash flow should be well protected in short term. In long term, its cash flow may be adversely affected due to lasting low commodity price or change in the supply and demand relation for midstream services. However, compared to upstream and downstream segments, Kinder Morgan has way smaller exposure to such risks. 

\section{Outlook}
   
Kinder Morgan owes ~41.5 billion USD debts and pays ~2 billion USD every year for interest cost. Historically, the company depended on the financial market to fund its growth. With such a big amount of debts, the company needs to re-finance its debts at maturity. A bad credit rating or an adverse financial market, will restrict the company from accessing the financial market, limit its ability to restructure its matured debts, and prevent it from borrowing money at a fair price. In 2015, its credit rating was downgraded below investment grade and its outlook was set to negative. Then the company decides to reduce dividend payments and high-grade its backlog. It targets a lower year-end debt to EBITDA ratio. It brings its credit rating back to Baa3 (stable). Previously, the priority of the company was growth. Then the focus has been shifted to strengthening balance sheet and improving credit rating. 

For 2016, Kinder Morgan expects to declare dividends of 0.50 per share. Its budgeted distributable cash flow available to common equity holders (i.e., after payment of preferred dividends) is approximately 4.7 billion USD and budgeted EBITDA is approximately 7.5 billion USD. Due to continued weakness in the energy sector in 2016, the company now expects EBITDA to be about 3 percent below its plan and distributable cash flow to be about 4 percent below its plan. The company expects to generate excess cash sufficient to fund its growth capital needs without needing to access capital markets and expects to achieve its targeted year-end debt to EBITDA ratio of 5.5 times (a single-digit decrease in EBITDA may make it very difficult to achieve this target, without dividend reduction, future growth project high-grading or asset divestment). Its growth capital forecast for 2016 is approximately 2.9 billion USD, a reduction of 400 million from its budget of approximately 3.3 billion and a reduction of 1.3 billion from its preliminary 2016 guidance of approximately 4.2 billion. 

\section{Short Term Risks}

Due to my lack of knowledge about debt restructuring, I am not sure how low credit rating and an adverse financial market will affect the debt re-financing of Kinder Morgan. I consider this as my primary risk and it results from my lack of financial knowledge. I am not sure whether Kinder Morgan can achieve its targeted year-end debt to EBITDA ratio of 5.5 times. I do not know whether the company has to divest some assets to reach this goal.

\section{Good Management}

The management team focuses on stable fee-based assets that are core to North American energy infrastructure. It creates a long-life business that consistently generates a large amount of distributable cash flow. The company has focused in seeking attractive capital investment opportunities, both expansion and acquisition. The management team holds quite a large amount of share in the company and they put emphasis on cost control. Kinder Morgan is a good business and it is quite transparent to investors. Recently, the management team has shifted their focus to strengthening balance sheet and improving credit rating. 

The company uses its distributable cash flow and funds from financial market to support growth and pay dividends. In the fee-based business model, new assets with synergy effect generate a large amount of consistent cash flow. With a good credit rating in a good financial market, it is favourable to access the financial market to fund growth projects. That is what the company had been doing. If the financial market is under pressure or the credit rating is below investment grade, borrowing or refinancing at a favourable price can not be guaranteed. Its cash flow might also be affected by its debts. So now, it is rational to focus on strengthening the balance sheet, slashing dividends, high-grading growth projects, and reducing debt to EBITDA ratio. The management team has made rational decisions from the long-term perspective. 

\section{Kinder Morgan Stocks}
I need to study more about debt restructuring and credit rating. I should keep an eye on how Kinder Morgan achieves its targeted 2016 year-end debt to EBITDA ratio. This stock should be a long-term investing. I should consider buying at a price below 17 USD per share.

\end{document}
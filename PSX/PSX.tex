\documentclass[11pt]{article}
\usepackage{amsmath,amssymb,graphicx}
\usepackage[letterpaper ,left=2cm,top=2.25cm,bottom=2.25cm,right=2.5cm,nohead,nofoot]{geometry}
\usepackage{color}
\definecolor{LinkColor}{rgb}{0.256,0.439,0.588}
\definecolor{MenuColor}{rgb}{0.739,0.394,0.399}
\usepackage[colorlinks,breaklinks,urlcolor=LinkColor]{hyperref}
\newcommand{\menuItem}[1]{{\color{MenuColor}\textsf{#1}}}
\usepackage{bm}
\usepackage{bbold}
\renewcommand{\vec}[1]{\mathbf{#1}}
\DeclareMathOperator{\sgn}{sgn}
\usepackage[T1]{fontenc}
\usepackage{times}
\usepackage[scaled]{beramono}
\begin{document}

\title{Phillips 66}
\author{Jin Xu}
\date{\today}
\maketitle

\section{Investment Strategy}

If I were able to fully own a business, I definitely look for one that will consistently generates a large amount of cash. After owning it for a long time, I want to sell it at even a higher price. The investment return would be a combination of cash return and value increase of this business. From the perspective of business, the investment return is the only purpose of owning a business. 

The value of a business should be only determined by its earning capacity in future years. Free cash flow or its other forms should be the best measure of the earning capacity for a stable business. Free cash flow is calculated as {\bf EBIT(1 - Tax Rate) + Depreciation \& Amortization - change in Net Working Capital - Capital Expenditure}. Free cash flow can be distributed to our debt lenders and the business owner. The remaining cash can also be used for business investment, either developing growth projects to increase revenues, or increasing operating efficiency to maximize profit margin. However, it makes sense only if the investment can create more than one dollar for every dollar invested. The free cash flow attributable to business owners should be adjusted by interest payment. As in pipeline business, the widely adopted measure of earning capacity is the distributable cash flow. It is calculated as {\bf Net Income + Depreciation, Depletion \& Amortization - Maintenance Capital Expenditures}. 

The owner does not have to run the business by himself. However, it is necessary for him to fully understand the business. I love simplicity. I even consider simplicity as my religious belief. Actually, a good business can be very simple. As the owner for a long period of time, we need the business to consistently generate stable free cash flow. We can see the consistency from the nature of the business. It should also be confirmed by the track record of its operating history. We should also need to have a positive outlook for its value increase.

Very often, the owner hires a capable executive team to run the business. It is extremely important for the owner to hire good leaders for his company. I strongly believe effective leadership and good management play a way more important role than functional employees. Corporate culture is the spirit of a company as an entity, which helps improve production efficiency and avoid making bad decisions. Effective leaders put emphasis on it. Good management team should always make rational decisions and focus on the long-term value of the company. Also the management team should stay close with the owner and communicate candidly. I believe any environment can be a good opportunity as long as we are prepared and know how to take advantage of it. Good executive team should be vigilant. They should stay fearful when the market is booming, always preparing for the advent of the downturn. When the market bears, they should feel greedy because it’s a great opportunity for the company to grow fast at a cheaper price.

In reality, I cannot afford to fully own a good business. I am only able to buy shares of public companies. The stock market is a great platform to buy a tiny portion of a company. If a public company has 1 billion shares outstanding and I only own 100 shares. It simply means I own 1/10000000 portion of the company. I always keep it in mind that owning stocks equals to owning a portion of the company itself. From this philosophy, the stock value should be always associated with a company. However, the market price of the stock can be higher than its value or lower than its value. Thus, my investment strategy is to buy the stock of a good company at a lower-than-value price, own it for a long period, and sell it ultimately at a higher-than-value price. My investment return is the cash return during my holding and the increase in the stock price from my buying to selling.            
 
\section{Good Business}

The oil \& gas industry can be divided into three segments: upstream, midstream and downstream. Upstream produces crude oil and natural gas. This segment bears a large amount of market risk. The performance primarily depends on the global crude oil price. The crude oil price is very volatile by its nature. It is cyclical but extremely difficult for anyone to predict. It suddenly dropped from more than 100 USD per barrel in August, 2014 to less than 30 in early 2016. Midstream stores and transports crude oil and natural gas for processing and processed products to the market. Midstream is the most stable segment among three. It transports oil and gas for its clients or itself and charges them on a fee basis. Downstream refines crude oil and produces petroleum products, manufactures chemical products using products from the upstream, and markets and sells the processed products. The primary market risk is the crack spread. Crack spread is the price difference between crude oil and refined petroleum products. Crack spread varies but not as much as the crude oil price. To be short, midstream is the most stable segment, downstream follows, but upstream is totally volatile.

Phillips 66 is an independent company, from the spin-off of midstream and downstream assets of ConocoPhillips. It is an integrated entity of two most stable segments in the oil \& gas industry. It consists of four business segments: refining, chemicals, marketing \& specialties, and midstream. The four integrated segments make Phillips 66 very simple. It buys crude oil and natural gas from the producers, and then sells the processed products into the market. It makes profit out of the price difference between the crude oil and processed products. 

The company now invests on increasing the efficiency of the downstream operations, organically growing the chemical segment, and growing integrated transportation midstream.    

\section{Free Cash Flow}
In the three-year period from 2013 to 2015, The midstream generates available cash flow 1.4 billion USD every year. As the company now focuses on growing this segment, we could expect cash flow from this segment increase year by year. The marketing \& specialties segment generates free cash flow 0.9 billion USD every year and it is quite stable year over year. Chemicals also generates free cash flow 0.9 billion USD per year. The company expects the global chemicals demand grow and the production cost decrease. It is investing on two major growth projects, whose planned startup is in the middle of 2017. The refining segment is a little bit more volatile but still generates a large amount of free cash flow, 2.0 billion USD every year. Refining margin primarily depends on the crack spread. It makes sense that the company only invests a small portion of money on this segment, increasing operating efficiency. It utilizes a large of money, investing on fee-based midstream assets. The corporate expenses are around 0.44 billion per year. In conclusion, over three years from 2013 to 2015, the free cash flow attributable to shareholders is 4.7 billion USD. It distributes around 40% of the generated free cash to shareholders through dividend and share buy-back. It re-invests the rest 60%. 

\section{Outlook}
Phillips 66 currently has tons of growth opportunities in the Chemicals and Midstream, especially in the fee-based transportation. Phillips 66 only carries less than 9 billion USD long-term debt. Considering it generates around 4.7 billion free cash each year, Phillips 66 is in a very good financial position. We can expect that Phillips 66 will increase its free cash flow to a great degree in the next few years. Even if a financial crisis happens, Phillips 66 is in a good position to easily weather this kind of storm.

I believe the demand for petroleum products is still increasing for the next 20 years. One concern is that government would add environmental charges on the carbon industry. However, I believe companies in this business sector could transfer the charge to its customers.     

\section{Good management}
The management team gives simple investor presentations, which helps investors better understand the business of Phillips 66. The strategy of Phillips 66 is rational, increasing its efficiency for the ongoing operations, invest on new growth projects. Its investment on Chemicals is based on increasing demand for petro-chemical products and its decreasing production cost. I like Phillips investing on midstream assets, especially on those integrable to its other three business segments. The majority of its investments are funded by its remaining free cash after distributing to shareholders. This helps the company stay in a good financial position even if its position is already very good. It guarantees that Phillips 66 can still do well in a financial crisis.  

\section{Phillips 66 stocks}
I should keep studying the midstream and downstream of the oil \& gas industry. I should also refer to its peers for more insights. I should keep buying Phillips 66 when the price is lower than 100 USD per share. In addition, in the case when a financial crisis strikes, Phillips 66 is definitely a good buy option. I do not see any reason for selling at a price lower than 150 USD per share. 

\end{document}